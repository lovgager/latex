\documentclass[12pt]{article} 
\usepackage[russian]{babel}
\usepackage{amsmath}			% for math formulas
\usepackage{amsthm}
\usepackage{setspace}
\usepackage{amsfonts}			% for math formulas
\usepackage{amssymb}
\usepackage[unicode, pdftex]{hyperref}
\usepackage[left=25mm, top=20mm, right=25mm, bottom=20mm, nohead, nofoot]{geometry}
\pagestyle{empty}
    \DeclareMathOperator*\uplim{\overline{lim}}

\begin{document}

%-------------------------------
%	TITLE SECTION
%-------------------------------


\begin{flushleft}
\url{https://www.facebook.com/profile.php?id=10000654912}
\end{flushleft}
\hrule 
\begin{flushright}
04.04.2021
\end{flushright}
\bigskip

%-------------------------------
%	CONTENTS
%-------------------------------

\newtheorem*{task}{Задача}
\begin{task}
Найти область сходимости вещественного степенного ряда
\begin{equation} \label{series}
    \sum\limits_{n=0}^{\infty} \frac{3^n\, x^{2n}}{n+2}.
\end{equation}
\end{task}

\noindent\textbf{Решение.} По теореме Коши--Адамара любой степенной ряд вида 
\begin{equation*}
    \sum\limits_{m=0}^{\infty} a_m x^m,
\end{equation*}
где $a_m$ -- вещественные числа, сходится при любых $x$, принадлежащих интервалу $(-R, R)$, и расходится при всех $x$, что $|x| > R$. Радиус сходимости $R$ можно найти из формулы
\begin{equation}\label{radius}
    \frac{1}{R} = \uplim_{m\to\infty} \sqrt[m]{|a_m|}.
\end{equation}
Здесь $\overline{\lim}$ обозначает верхний предел, то есть максимальный предел любой вложенной подпоследовательности.

Так как в исходный степенной ряд входят только чётные степени переменной $x$, то $a_m = 0$ для всех нечётных $m$, а для чётных можно записать
\begin{equation*}
    a_{2n} = \frac{3^n}{n + 2}.
\end{equation*}
Тогда в общем виде последовательность $\{a_m\}$ определяется следующим образом:
\begin{equation*}
    a_m = 
    \begin{cases}
        \enskip 0, & m = 2n - 1, \vspace{2mm} \\
        \enskip\displaystyle\frac{3^{m/2}}{\frac{m}{2} + 2}, & m = 2n.
    \end{cases}
\end{equation*}
Отсюда видно, что её можно разбить на две подпоследовательности: первая состоит только из нулей, а предел второй неотрицателен (т. к. все её элементы положительны), поэтому верхним пределом в формуле (\ref{radius}) является следующее число:
\begin{equation*}
    \lim_{m\to\infty} \sqrt[m]{\frac{3^{m/2}}{\frac{m}{2} + 2}} = \lim_{m\to\infty} \frac{\sqrt{3}}{\left(\frac{m}{2} + 2\right)^{1/m}} = \sqrt{3}.
\end{equation*}
Здесь мы воспользовались тем фактом (который следует отдельно доказать!), что
\begin{equation*}
    \lim_{m\to\infty} \left(bm + c\right)^{1/m} = 1
\end{equation*}
для любых положительных чисел $b, c$. Таким образом, из формулы (\ref{radius}) получаем, что
\begin{equation*}
    R = \frac{1}{\sqrt{3}}.
\end{equation*}
Так как в теореме Коши--Адамара ничего не утверждается о сходимости степенного ряда на границах интервала $(-R, R)$, этот вопрос разберём отдельно. Подставим в (\ref{series}) значение $x = 1/\sqrt{3}:$
\begin{equation*}
    \sum\limits_{n=0}^{\infty} \frac{3^n\, (1/\sqrt{3})^{2n}}{n+2} = \sum\limits_{n=0}^{\infty} \frac{3^n\, (1/3)^n}{n+2} = \sum\limits_{n=0}^{\infty} \frac{1}{n+2} = \infty.
\end{equation*}
Получили гармонический числовой ряд, то есть ряд вида
\begin{equation*}
    1 + \frac{1}{2} + \frac{1}{3} + \frac{1}{4} + ...\,,
\end{equation*}
который, как известно, расходится. Случай $x = -1/\sqrt{3}$ рассматривается точно так же. Итоговый ответ: областью сходимости ряда (\ref{series}) является интервал без граничных точек
\begin{equation*}
    \left(-\frac{1}{\sqrt{3}}, \frac{1}{\sqrt{3}} \right).
\end{equation*}

\end{document}