\documentclass[12pt]{article} 
\usepackage[russian]{babel}
\usepackage{amsmath}			% for math formulas
\usepackage{amsthm}
\usepackage{setspace}
\usepackage{amsfonts}			% for math formulas
\usepackage{amssymb}
\usepackage[unicode, pdftex]{hyperref}
\usepackage[left=25mm, top=20mm, right=25mm, bottom=20mm, nohead, nofoot]{geometry}
\usepackage{graphicx,xcolor}
\pagestyle{empty}

\begin{document}

%-------------------------------
%	TITLE SECTION
%-------------------------------


\begin{flushleft}
\url{https://www.facebook.com/profile.php?id=10000654912}
\end{flushleft}
\hrule 
\begin{flushright}
03.04.2021
\end{flushright}
\bigskip

%-------------------------------
%	CONTENTS
%-------------------------------

\newtheorem*{task}{Задача}
\begin{task}
Доказать, что в прямоугольной декартовой системе координат на плоскости расстояние $d$ от точки $M_0 = (x_0, y_0)$ до прямой $\alpha: Ax+By+C=0, \enskip \sqrt{A^2+B^2} \neq 0,$ вычисляется по формуле
\begin{equation*}
    d = \frac{|Ax_0+By_0+C|}{\sqrt{A^2+B^2}}.
\end{equation*}
\end{task}

\noindent\textbf{Решение.} Проведём через точку $M_0$ прямую $\beta$, перпендикулярную прямой $\alpha$. Обозначим точку их пересечения через $D$.
\begin{figure}[ht]\centering
    \def\svgwidth{7cm}
    \input{drawing.pdf_tex}
\end{figure}

\noindent Тогда направляющий вектор прямой $\beta$ совпадает с вектором нормали прямой $\alpha$ и равен $\vec{n} = \{A,B\}$ (он не обязательно направлен в ту же полуплоскость, в которой лежит точка $M_0$, т. к. рисунок лишь схематичный). Значит, вектор $\overrightarrow{DM_0}$ сонаправлен с вектором нормали, то есть справедливо равенство
\begin{equation*}
    \overrightarrow{DM_0} = t\vec{n}
\end{equation*}
для некоторого числа $t$. Пусть $D = (x_D, y_D)$, тогда $M_0 = D + t\vec{n}$, а в покоординатной записи 
\begin{equation*}
    x_0 = x_D + At, \enskip y_0 = y_D + Bt,
\end{equation*}
\begin{equation}
    \label{1}x_D = x_0 - At, \enskip y_D = y_0 - Bt.
\end{equation}
Так как точка $D$ принадлежит прямой $\alpha$, для её координат справедливо равенство $Ax_D + By_D + C$. Подставляя сюда выражения (\ref{1}), получим
\begin{gather*}
    A(x_0 - At) + B(y_0 - Bt) + C = 0, \\
    Ax_0 + By_0+C = t(A^2 + B^2), \\
    t = \frac{Ax_0+By_0+C} {A^2 + B^2}.
\end{gather*}
Искомое расстояние $d$, очевидно равно длине вектора $\overrightarrow{DM_0}$. В итоге получаем
\begin{equation*}
    d = |\overrightarrow{DM_0}| = |t|\cdot|\vec{n}|
    = \frac{Ax_0+By_0+C} {A^2 + B^2} \cdot \sqrt{A^2 + B^2} = \frac{|Ax_0+By_0+C|}{\sqrt{A^2+B^2}}.
\end{equation*}

\end{document}