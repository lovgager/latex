\documentclass[12pt]{article} 
\usepackage[russian]{babel}
\usepackage{amsmath}			% for math formulas
\usepackage{amsthm}
\usepackage{setspace}
\usepackage{amsfonts}			% for math formulas
\usepackage{amssymb}
\usepackage[unicode, pdftex]{hyperref}
\usepackage[left=25mm, top=20mm, right=25mm, bottom=20mm, nohead, nofoot]{geometry}
\pagestyle{empty}
\begin{document}

%-------------------------------
%	TITLE SECTION
%-------------------------------


\begin{flushleft}
\url{https://www.facebook.com/profile.php?id=10000654912}

\url{https://github.com/lovgager/latex}
\end{flushleft}
\hrule 
\begin{flushright}
25.04.2021
\end{flushright}
\bigskip

%-------------------------------
%	CONTENTS
%-------------------------------

\newtheorem*{task}{Задача}
\begin{task}
По выборке из $n$ независимых нормально распределённых случайных величин построить доверительный интервал уровня $\gamma$ для неизвестного математического ожидания в случае известной дисперсии.
\end{task}

\noindent\textbf{Решение.} 
Обозначим случайные величины в выборке через $X_1, ..., X_n$. Пусть $\sigma_0$ -- известное стандартное отклонение (квадратный корень из дисперсии), $\overline{X}$ -- выборочное среднее, то есть
\begin{equation*}
    \overline{X} = \frac{1}{n} \sum\limits_{i=1}^{n}X_i\,.
\end{equation*}
Это точечная оценка для неизвестного математического ожидания $a$. Но нам нужно построить доверительный интервал. Для этого можно рассмотреть случайную величину
\begin{equation*}
    Z = \frac{\overline{X} - a}{\sigma_0}\sqrt{n}.
\end{equation*}
Можно убедиться в том, что она имеет стандартное нормальное распределение. Действительно, суммирование нормальных величин, добавление константы или умножение на число не портят нормальность, а из независимости случайных величин $X_1, ..., X_n$ и линейности математического ожидания и дисперсии следует, что $\mathbb{E}Z = 0, \enskip \mathbb{D}Z = 1$, то есть $Z \sim \mathcal N(0, 1)$. Таким образом, эта случайная величина задаётся в зависимости от неизвестного параметра $a$, но её распределение от него не зависит. Мы можем вычислить вероятность принадлежности этой величины любому заданному интервалу $(b, c)$:
\begin{equation*}
    \mathbb{P}(b < Z < c) = \Phi(c) - \Phi(b),
\end{equation*}
где $\Phi(x)$ -- функция стандартного нормального распределения. Возьмём симметричный интервал $(-c, c), \enskip c > 0,$ и приравняем вероятность к $\gamma$:
\begin{equation*}
    \gamma = \mathbb{P}(-c < Z < c) = \Phi(c) - \Phi(-c) = 2\Phi(c) - 1.
\end{equation*}
Здесь мы использовали свойство о том, что $\Phi(-x) = 1 - \Phi(x)$ (объясняется оно тем, что стандартное нормальное распределение симметрично, поэтому вероятность оказаться левее $-c$ -- это всё равно что вероятность попасть правее $c$). Из равенства $\gamma = 2\Phi(c) - 1$ следует, что
\begin{equation*}
    c = \Phi^{-1}\left(\frac{1 + \gamma}{2}\right).
\end{equation*}
Здесь $\Phi^{-1}(y)$ -- обратная функция к $\Phi(x)$ или квантиль стандартного нормального распределения, то есть значение, которое стандартная нормальная случайная величина не превышает с вероятностью $y$. 

Вернёмся к равенству для вероятности:
\begin{equation*}
    \gamma = \mathbb{P}(-c < Z < c) \quad\Leftrightarrow\quad
    \gamma = \mathbb{P}\left(-c < \frac{\overline{X} - a}{\sigma_0} \sqrt{n} < c\right) \quad\Leftrightarrow\quad 
\end{equation*}
\begin{equation*}\vspace{5mm}
    \quad\Leftrightarrow\quad \gamma = \mathbb{P}\left(\overline{X} - \frac{\sigma_0}{\sqrt{n}}\,c < a < \overline{X} + \frac{\sigma_0}{\sqrt{n}}\,c\right).
\end{equation*}
Число $c$ известно и зависит от $\gamma$. Получаем интервал, границы которого являются случайными величинами и который накрывает неизвестное число $a$ с наперёд заданной вероятностью $\gamma$:\vspace{3mm}
\begin{equation*}
    \left(\overline{X} - \frac{\sigma_0}{\sqrt{n}}\,c,\; \overline{X} + \frac{\sigma_0}{\sqrt{n}}\,c\right), \quad c = \Phi^{-1}\left(\frac{1 + \gamma}{2}\right).
\end{equation*}

\end{document}