\documentclass[12pt]{article} 
\usepackage[russian]{babel}
\usepackage{amsmath}			% for math formulas
\usepackage{amsthm}
\usepackage{setspace}
\usepackage{amsfonts}			% for math formulas
\usepackage{amssymb}
\usepackage[unicode, pdftex]{hyperref}
\usepackage[left=25mm, top=20mm, right=25mm, bottom=20mm, nohead, nofoot]{geometry}
\pagestyle{empty}
\begin{document}

%-------------------------------
%	TITLE SECTION
%-------------------------------


\begin{flushleft}
\url{https://www.facebook.com/profile.php?id=10000654912}

\url{https://github.com/lovgager/latex}
\end{flushleft}
\hrule 
\begin{flushright}
12.04.2021
\end{flushright}
\bigskip

%-------------------------------
%	CONTENTS
%-------------------------------

\newtheorem*{task}{Задача}
\begin{task}
Аппроксимировать вторую производную $y''(0)$ функции $y(x)$ в точке $x=0$ разностным отношением, используя значения $y(x)$ в узлах трёхточечного шаблона $x_1=-h/4, \enskip x_2=0, \enskip x_3=3h/4.$ Вычислить значение разностного отошения (приближённое значение $y''(0)$) для $y(x)=x^3$ при $h=0.04$.
\end{task}

\noindent\textbf{Решение.} 
Искомое разностное отношение имеет вид:
\begin{equation}\label{appr}
    y''(0) \approx ay(x_1)+by(x_2)+cy(x_3) \equiv ay(-h/4) + by(0) + cy(3h/4),
\end{equation}
где $a,b,c$ -- неизвестные числа, зависящие от $h$. В правой части этой формулы разложим все $y$ в ряд Тейлора в окрестности точки 0, предполагая, что производные всех необходимых порядков в точке 0 существуют:
\begin{gather*}
    y(x) = \sum\limits_{n=0}^{\infty} \frac{y^{(n)}(0)}{n!} x^n \equiv y(0) + y'(0)\,x + \frac{y''(0)}{2}\,x^2 + O(x^3),\\
    y(-h/4) = y(0) + y'(0)\,\frac{h}{4} + y''(0)\,\frac{h^2}{32} + O(h^3),\\\\
    y(3h/4) = y(0) + y'(0)\,\frac{3h}{4} + y''(0)\,\frac{9h^2}{32} + O(h^3).
\end{gather*}
Подставляем в формулу (\ref{appr}):
\begin{equation*}
    y''(0) \approx a\left(y(0) + y'(0)\,\frac{h}{4} + y''(0)\,\frac{h^2}{32}\right) + by(0) + c\left(y(0) + y'(0)\,\frac{3h}{4} + y''(0)\,\frac{9h^2}{32}\right) + O(h^3).
\end{equation*}
Мы можем обеспечить третий порядок аппроксимации по $h$, если приравняем к нулю коэффициенты при $y(0)$ и $y'(0)$, а коэффициент при $y''(0)$ -- к единице:
\begin{equation*}
    \begin{cases}
    a + b + c = 0, \vspace{2mm}\\
    \displaystyle\frac{ah}{4} + \frac{3ch}{4} = 0, \vspace{3mm}\\
    \displaystyle\frac{ah^2}{32} + \frac{9ch^2}{32} = 1.
    \end{cases}
\end{equation*}
Решение системы:
\begin{equation*}
    a = -\,\frac{16}{h^2}, \quad b = \frac{32}{3h^2}, \quad c = \frac{16}{3h^2}.
\end{equation*}
\newpage
Таким образом, искомая формула имеет вид:
\begin{equation*}
    y''(0) \approx \frac{-48y(-h/4) + 32y(0) + 16y(3h/4)}{3h^2}.
\end{equation*}
Более точно, как мы показали:
\begin{equation*}
    y''(0) = \frac{-48y(-h/4) + 32y(0) + 16y(3h/4)}{3h^2} + O(h^3).
\end{equation*}

Пусть теперь $y(x) = x^3, \enskip h = 0.04$. Тогда из этой формулы получаем
\begin{equation*}
    y''(0) \approx \frac{-48(-0.01)^3 + 32\cdot 0 + 16(0.03)^3}{3(0.01)^3} = 0.016.
\end{equation*}
Точный результат равен $y''(0) = 0$, поэтому значение $0.016$ можно считать неплохим приближением.
\end{document}