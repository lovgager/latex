\documentclass[12pt]{article} 
\usepackage[russian]{babel}
\usepackage{amsmath}			% for math formulas
\usepackage{amsthm}
\usepackage{setspace}
\usepackage{amsfonts}			% for math formulas
\usepackage{amssymb}
\usepackage[unicode, pdftex]{hyperref}
\usepackage[left=25mm, top=20mm, right=25mm, bottom=20mm, nohead, nofoot]{geometry}
\usepackage{graphicx,xcolor}
\pagestyle{empty}

\begin{document}

%-------------------------------
%	TITLE SECTION
%-------------------------------


\begin{flushleft}
\url{https://www.facebook.com/profile.php?id=10000654912}
\url{https://github.com/lovgager/latex}
\end{flushleft}
\hrule 
\begin{flushright}
15.04.2021
\end{flushright}
\bigskip

%-------------------------------
%	CONTENTS
%-------------------------------

\newtheorem*{task}{Задача}
\begin{task}
На плоскости прямоугольных декартовых координат $(x,y)$ расположен эллипс с центром в начале координат и полуосями $1/\sqrt{3}, \enskip 1/3$. Известно, что большая ось эллипса образует угол $\alpha = \arccos{(1/\sqrt{3})}$ с осью абсцисс (угол отсчитывается от оси абсцисс против часовой стрелки). Найти уравнение эллипса в координатах $(x,y).$
\end{task}

\noindent\textbf{Решение.} Введём прямоугольную декартову систему координат $(u,v)$, в которой эллипс имеет канонический вид, то есть такой, что его оси совпадают с осями координат:
\begin{figure}[ht]\centering
    \def\svgwidth{7cm}
    \input{drawing.pdf_tex}
\end{figure}

\noindent В этих координатах его можно задать каноническим уравнением 
\begin{equation*}
    \frac{u^2}{a^2} + \frac{v^2}{b^2} = 1,
\end{equation*}
где $a,b$ -- большая и малая полуоси соответственно. В данном случае $a=1/\sqrt{3}, \enskip b=1/3$ по условию. Поэтому уравнение имеет вид:
\begin{equation}\label{canon_eq}
    3u^2+9v^2=1.
\end{equation}
Теперь нужно перейти от координат $(u,v)$ к координатам $(x,y)$. Так как эллипс поворачивается на угол $\alpha$ против часовой стрелки, то координаты $(x,y)$ получены путём вращения $(u,v)$ на тот же угол по часовой стрелке:
\begin{figure}[ht]\centering
    \def\svgwidth{7cm}
    \input{drawing2.pdf_tex}
\end{figure}

\newpage\noindent Пунктирную линию на рисунке выше следует понимать как ось $u$ в старой системе координат. Из рисунка видно, например, что вектор $(1,0)$ в старой системе координат переходит в вектор $(\cos\alpha, \sin\alpha)$. Вообще, можно записать, что переход к новой системе координат осуществляется с помощью матрицы поворота следующим образом: 
\begin{equation*}
    \begin{bmatrix}
    x\\y
    \end{bmatrix}
    =
    \begin{bmatrix}
    \cos\alpha & -\sin\alpha \\
    \sin\alpha & \cos\alpha
    \end{bmatrix}
    \cdot
    \begin{bmatrix}
    u\\v
    \end{bmatrix}.
\end{equation*}
Отсюда легко можно выразить $u,v$, поскольку любая матрица поворота является ортогональной, то есть её обратная матрица совпадает с транспонированной:
\begin{equation*}
    \begin{bmatrix}
    u\\v
    \end{bmatrix}
    =
    \begin{bmatrix}
    \cos\alpha & \sin\alpha \\
    -\sin\alpha & \cos\alpha
    \end{bmatrix}
    \cdot
    \begin{bmatrix}
    x\\y
    \end{bmatrix}.
\end{equation*}
Подставляем выражения для $u,v$ в уравнение (\ref{canon_eq}) и учитываем, что $\cos\alpha = 1/\sqrt{3}$, а следовательно, $\sin\alpha = \sqrt{2}/\sqrt{3}$:
\begin{equation*}
    3(x\cos\alpha + y\sin\alpha)^2 + 9(-x\sin\alpha + y\cos\alpha) = 1,
\end{equation*}
\begin{equation*}
    3\left(\frac{x}{\sqrt{3}} + \frac{y\sqrt{2}}{\sqrt{3}}\right)^2 + 9\left(\frac{-x\sqrt{2}}{\sqrt{3}} + \frac{y}{\sqrt{3}}\right) = 1,
\end{equation*} \vspace{2mm}
\begin{equation*}
    x^2 + 2\sqrt{2}xy + 2y^2 + 3(2x^2 - 2\sqrt{2}xy + y^2) = 1,
\end{equation*}
\begin{equation*}
    7x^2 - 4\sqrt{2}xy + 5y^2 = 1.
\end{equation*}
Получили искомое уравнение эллипса. Отметим, что эта задача в частном случае является обратной к задаче приведения квадратичной формы к каноническому виду, которая может быть решена методом Лагранжа.

\end{document}