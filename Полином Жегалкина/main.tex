\documentclass[12pt]{article} 
\usepackage[russian]{babel}
\usepackage{amsmath}			% for math formulas
\usepackage{amsthm}
\usepackage{setspace}
\usepackage{amsfonts}			% for math formulas
\usepackage{amssymb}
\usepackage[unicode, pdftex]{hyperref}
\usepackage[left=25mm, top=20mm, right=25mm, bottom=20mm, nohead, nofoot]{geometry}
\pagestyle{empty}
\begin{document}

%-------------------------------
%	TITLE SECTION
%-------------------------------


\begin{flushleft}
\url{https://www.facebook.com/profile.php?id=10000654912}

\url{https://github.com/lovgager/latex}
\end{flushleft}
\hrule 
\begin{flushright}
22.04.2021
\end{flushright}
\bigskip

%-------------------------------
%	CONTENTS
%-------------------------------

\newtheorem*{task}{Задача}
\begin{task}
Методом неопределённых коэффициентов найти полином Жегалкина для функции алгебры логики $f(x_1, x_2, x_3)$, заданной таблицей значений:
\begin{table}[h]
    \centering
    \begin{tabular}{|c|c|c|c|}\hline
         $x_1$ & $x_2$ & $x_3$ & $f$ \\ \hline
         0 & 0 & 0 & 0 \\
         0 & 0 & 1 & 0 \\
         0 & 1 & 0 & 0 \\
         0 & 1 & 1 & 0 \\
         1 & 0 & 0 & 0 \\
         1 & 0 & 1 & 1 \\
         1 & 1 & 0 & 1 \\
         1 & 1 & 1 & 1 \\\hline
    \end{tabular}
\end{table}
\end{task}

\noindent\textbf{Решение.} 
Полином Жегалкина ищется в виде
\begin{equation*}
    f(x_1,x_2,x_3) = c_0 \oplus c_1x_1 \oplus c_2x_2 \oplus c_3x_3 \oplus c_{12}x_1x_2 \oplus c_{13}x_1x_3 \oplus c_{23}x_2x_3 \oplus c_{123}x_1x_2x_3,
\end{equation*}
где $\oplus$ обозначает логическое "ИЛИ" (сложение по модулю 2), а $c_i \in \{0, 1\}$ -- неизвестные коэффициенты, подлежащие определению. Будем подставлять последовательно каждую строку значений из таблицы в это выражение и находить коэффициенты. Например,
\begin{gather*}
    f(0,0,0) = 0 = c_0,\\
    f(0,0,1) = 0 = c_0 \oplus c_3=0\quad\Rightarrow c_3=0,\\
    f(0,1,0) = 0 = c_0 \oplus c_2=0\quad\Rightarrow c_2=0,\\ ...
\end{gather*}
В итоге получаем
\begin{equation*}
    c_{12} = c_{13} = c_{123} = 1,
\end{equation*}
остальные коэффициенты равны нулю. Ответ:
\begin{equation*}
    f(x_1, x_2, x_3) = x_1x_2 \oplus x_1x_3 \oplus x_1x_2x_3.
\end{equation*}
\end{document}