\documentclass[12pt]{article} 
\usepackage[russian]{babel}
\usepackage{amsmath}			% for math formulas
\usepackage{amsthm}
\usepackage{setspace}
\usepackage{amsfonts}			% for math formulas
\usepackage{amssymb}
\usepackage[unicode, pdftex]{hyperref}
\usepackage[left=25mm, top=20mm, right=25mm, bottom=20mm, nohead, nofoot]{geometry}
\pagestyle{empty}
\begin{document}

%-------------------------------
%	TITLE SECTION
%-------------------------------


\begin{flushleft}
\url{https://www.facebook.com/profile.php?id=10000654912}

\url{https://github.com/lovgager/latex}
\end{flushleft}
\hrule 
\begin{flushright}
05.04.2021
\end{flushright}
\bigskip

%-------------------------------
%	CONTENTS
%-------------------------------

\newtheorem*{task}{Задача}
\begin{task}
Проверить, что данное уравнение является уравнением в полных дифференциалах, и решить его.
\begin{equation} \label{task_eq}
    \left(\frac{x}{\sin{y}}+2\right)dx + \frac{(x^2+1) \cos{y}}{\cos{2y}-1}dy = 0. 
\end{equation}
\end{task}

\noindent\textbf{Решение.} Напомним, что уравнение
\begin{equation}\label{main_eq}
    M(x,y)dx + N(x,y)dy = 0
\end{equation}
называется уравнением в полных дифференциалах, если его левая часть представляет собой полный дифференциал некоторой функции $F(x,y),$ то есть 
\begin{equation*}
    dF(x,y) = M(x,y)dx + N(x,y)dy.
\end{equation*}
Это имеет место, если выполнены равенства
\begin{equation}\label{partial_der}
    \frac{\partial F}{\partial x} = M(x,y), \quad \frac{\partial F}{\partial y} = N(x,y).
\end{equation}
Другими словами, для решения уравнения (\ref{main_eq}) достаточно найти функцию $F(x,y),$ зная её частные производные $F_x' = M(x,y)$ и $F_y' = N(x,y).$ 

Для достаточно гладких функций справедливо равенство вторых частных производных, то есть
\begin{equation*}
    \frac{\partial^2F}{\partial x\partial y} = \frac{\partial^2F}{\partial y\partial x}.
\end{equation*}
Отсюда и из равенств (\ref{partial_der}) следует, что
\begin{equation}\label{condition}
    \frac{\partial M}{\partial y} = \frac{\partial N}{\partial x}.
\end{equation}
Это необходимое условие на то, чтобы уравнение (\ref{main_eq}) являлось уравнением в полных дифференциалах. Вернёмся к уравнению (\ref{task_eq}) и проверим для него условие (\ref{condition}).
\begin{equation*}
    M(x,y) = \frac{x}{\sin{y}}+2, \quad\quad N(x,y) = \frac{(x^2+1) \cos{y}}{\cos{2y}-1},
\end{equation*}
\begin{equation*}
    \frac{\partial M}{\partial y} = -x\,\frac{\cos{y}}{\sin^2 y}, \quad\quad \frac{\partial N}{\partial x} = 2x\,\frac{\cos{y}}{\cos{2y}-1} = -x\,\frac{\cos{y}}{\sin^2 y}.
\end{equation*}
\newpage
Таким образом, условие (\ref{condition}) выполнено. Для нахождения функции $F(x,y)$ можно проинтегрировать первое из уравнений (\ref{partial_der}) по $x$, при этом считаем $y$ постоянным:
\begin{equation*}
    F(x,y) = \int M(x,y)dx = \int \left(\frac{x}{\sin{y}}+2\right)dx = \frac{x^2}{2\sin{y}} + 2x + \varphi(y).
\end{equation*}
Для нахождения неизвестной функции $\varphi(y)$ подставим это выражение во второе из уравнений (\ref{partial_der}):
\begin{equation*}
    \left(\frac{x^2}{2\sin{y}} + 2x + \varphi(y)\right)_y' = \frac{(x^2+1) \cos{y}}{\cos{2y}-1},\vspace{3mm}
\end{equation*}
\begin{equation*}
    -\frac{x^2\cos{y}}{2\sin^2 y} + \varphi'(y) = -\frac{(x^2+1)\cos{y}}{2\sin^2 y},\vspace{3mm}
\end{equation*}
\begin{equation*}
    \varphi'(y) = -\frac{\cos{y}}{2\sin^2 y},\vspace{3mm}
\end{equation*}
\begin{equation*}
    \varphi(y) = \frac{1}{2\sin{y}} + \text{const}.
\end{equation*}
Следовательно, можно взять
\begin{equation*}
    F(x,y) = \frac{x^2 + 1}{2\sin{y}} + 2x,
\end{equation*}
а общее решение уравнения (\ref{task_eq}) имеет вид
\begin{equation*}
    \frac{x^2 + 1}{2\sin{y}} + 2x = C.
\end{equation*}
\end{document}