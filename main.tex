\documentclass[12pt]{article} 
\usepackage[russian]{babel}
\usepackage{amsmath}			% for math formulas
\usepackage{amsthm}
\usepackage{setspace}
\usepackage{amsfonts}			% for math formulas
\usepackage{amssymb}
\usepackage[unicode, pdftex]{hyperref}
\usepackage[left=25mm, top=20mm, right=25mm, bottom=20mm, nohead, nofoot]{geometry}
\pagestyle{empty}

\begin{document}

%-------------------------------
%	TITLE SECTION
%-------------------------------


\begin{flushleft}
\url{https://www.facebook.com/profile.php?id=10000654912}
\end{flushleft}
\hrule 
\begin{flushright}
02.04.2021
\end{flushright}
\bigskip

%-------------------------------
%	CONTENTS
%-------------------------------

\newtheorem*{task}{Задача}
\begin{task}
Привести квадратичную форму 
\begin{equation*}
    f(x_1,x_2,x_3)=2x_1^2 + 8x_1x_2 + 4x_1x_3 + 9x_2^2 + 19x_3^2
\end{equation*}
к каноническому виду с помощью метода Лагранжа. Выписать преобразование координат, осуществляющее такое приведение.
\end{task}

\noindent\textbf{Решение.} Метод Лагранжа (метод выделения полных квадратов) состоит в последовательном выделении полных квадратов сначала в группе слагаемых, содержащих $x_1$, затем содержащих $x_2$ и т. д. Имеем
\begin{gather*}
    f = (2x_1^2 + 8x_1x_2 + 4x_1x_3) + 9x_2^2 + 19x_3^2 = \\
    = 2(x_1^2 + 2x_1(2x_2 + x_3) + (2x_2 + x_3)^2) - 2(2x_2+x_3)^2 + 9x^2 + 19x_3 = \\
    = 2(x_1 + 2x_2 + x_3)^2 - 8x_2^2 - 2x_3^2 - 8x_2x_3 + 9x_2^2 + 19x_3^2 = \\ 
    = 2(x_1 + 2x_2 + x_3)^2 + (x_2 - 4x_3)^2 - 16x_3^2 + 17x_3^2 = \\ 
    = 2(x_1 + 2x_2 + x_3)^2 + (x_2 - 4x_3)^2 + x_3^2 = 2y_1^2 + y_2^2 + y_3^2,
\end{gather*}
где $y_1 = x_1 + 2x_2 + x_3, y_2 = x_2 - 4x_3, y_3 = x_3.$ Остаётся выразить $x_1, x_2, x_3$. Окончательно формулы преобразования координат имеют вид

\begin{equation*}
\begin{cases}
    \enskip x_1 = y_1 - 2y_2 - 9y_3, \\
    \enskip x_2 = y_2 + 4y_3, \\
    \enskip x_3 = y_3. \\
\end{cases}
\end{equation*}


\end{document}