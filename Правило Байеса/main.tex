\documentclass[12pt]{article} 
\usepackage[russian]{babel}
\usepackage{amsmath}			% for math formulas
\usepackage{amsthm}
\usepackage{setspace}
\usepackage{amsfonts}			% for math formulas
\usepackage{amssymb}
\usepackage[unicode, pdftex]{hyperref}
\usepackage[left=25mm, top=20mm, right=25mm, bottom=20mm, nohead, nofoot]{geometry}
\pagestyle{empty}
\begin{document}

%-------------------------------
%	TITLE SECTION
%-------------------------------


\begin{flushleft}
\url{https://www.facebook.com/profile.php?id=10000654912}

\url{https://github.com/lovgager/latex}
\end{flushleft}
\hrule 
\begin{flushright}
18.04.2021
\end{flushright}
\bigskip

%-------------------------------
%	CONTENTS
%-------------------------------

\newtheorem*{task}{Задача}
\begin{task}
Вероятность того, что книга лежит в шкафу, равна $p$. Если книга в шкафу, то она может с равной вероятностью лежать в одном из 4 ящиков. Был наугад открыт один из ящиков, и книги в нём не оказалось. Какова теперь вероятность, что книга лежит в шкафу?
\end{task}

\noindent\textbf{Решение.} 
Обозначим события: $A$ -- книга не лежит в выбранном ящике; $B$ -- книга лежит в шкафу. Нужно вычислить $\mathbb{P}(B|A).$ Используем формулу Байеса:
\begin{equation}\label{bayes}
    \mathbb{P}(B|A) = \frac{\mathbb{P}(A|B)\,\mathbb{P}(B)} {\mathbb{P}(A)}.
\end{equation}
По условию $\mathbb{P}(B) = p$. Если книга в шкафу, то она лежит с равной вероятностью в одном из четырёх ящиков, то есть $\mathbb{P}(A|B) = 3/4$. Вычислим $\mathbb{P}(A)$ по формуле полной вероятности (зная, что $B$ и $\overline{B}$ -- это полная группа событий):
\begin{equation*}
    \mathbb{P}(A) = \mathbb{P}(B)\,\mathbb{P}(A|B) + \mathbb{P}(\overline{B})\, \mathbb{P}(A|\overline{B}) = \frac{3}{4}p + 1 - p = 1 - \frac{p}{4}.
\end{equation*}
Здесь мы воспользовались тем, что $\mathbb{P}(\overline{B}) = 1 - \mathbb{P}(B)$ и $\mathbb{P}(A|\overline{B}) = 1$. Подставляем все вычисленные значения в формулу (\ref{bayes}):
\begin{equation*}
    \mathbb{P}(B|A) = \frac{\displaystyle\frac{3}{4}p}{1 - \displaystyle\frac{p}{4}} = \frac{3p}{4-p}.
\end{equation*}
\end{document}