\documentclass[12pt]{article} 
\usepackage[russian]{babel}
\usepackage{amsmath}			% for math formulas
\usepackage{amsthm}
\usepackage{setspace}
\usepackage{amsfonts}			% for math formulas
\usepackage{amssymb}
\usepackage[unicode, pdftex]{hyperref}
\usepackage[left=25mm, top=20mm, right=25mm, bottom=20mm, nohead, nofoot]{geometry}
\pagestyle{empty}
\begin{document}

%-------------------------------
%	TITLE SECTION
%-------------------------------


\begin{flushleft}
\url{https://www.facebook.com/profile.php?id=10000654912}

\url{https://github.com/lovgager/latex}
\end{flushleft}
\hrule 
\begin{flushright}
23.04.2021
\end{flushright}
\bigskip

%-------------------------------
%	CONTENTS
%-------------------------------

\newtheorem*{task}{Задача}
\begin{task}
Исследовать на дифференцируемость функцию комплексной переменной $z$:
\begin{equation*}
    f(z) \equiv f(x+iy) = x+iy^2.
\end{equation*}
\end{task}

\noindent\textbf{Решение.} 
Напомним утверждение из ТФКП: функция $f(z) = u(x,y) + iv(x,y)$ дифференцируема в точке $z_0=x_0 + iy_0$ тогда и только тогда, когда функции $u(x,y), v(x,y)$ дифференцируемы в точке $(x_0, y_0)$ и в этой точке выполняются \textit{условия Коши--Римана}:
\begin{equation}\label{CR}
    \frac{\partial u}{\partial x} = 
    \frac{\partial v}{\partial y}, \quad\quad
    \frac{\partial u}{\partial u} = 
    -\frac{\partial v}{\partial x}
\end{equation}

В нашей задаче $u(x,y) = x, v(x,y) = y^2$. Находим частные производные:
\begin{equation*}
    \frac{\partial u}{\partial x} = 1, \quad 
    \frac{\partial u}{\partial y} = 0, \quad 
    \frac{\partial v}{\partial x} = 0, \quad 
    \frac{\partial v}{\partial y} = 2y. \quad 
\end{equation*}
Дифференцируемость функций $u, v$ очевидна. Условия Коши--Римана (\ref{CR}) выполняются только для тех точек, для которых $y = 1/2$. Таким образом, функция $f(z)$ дифференцируема во всех точках вида $z = x + i/2, \enskip x\in\mathbb{R},$ и только в них.
\end{document}