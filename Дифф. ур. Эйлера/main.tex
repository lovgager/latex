\documentclass[12pt]{article} 
\usepackage[russian]{babel}
\usepackage{amsmath}			% for math formulas
\usepackage{amsthm}
\usepackage{setspace}
\usepackage{amsfonts}			% for math formulas
\usepackage{amssymb}
\usepackage[unicode, pdftex]{hyperref}
\usepackage[left=25mm, top=20mm, right=25mm, bottom=20mm, nohead, nofoot]{geometry}
\pagestyle{empty}
\begin{document}

%-------------------------------
%	TITLE SECTION
%-------------------------------


\begin{flushleft}
\url{https://www.facebook.com/profile.php?id=10000654912}

\url{https://github.com/lovgager/latex}
\end{flushleft}
\hrule 
\begin{flushright}
19.04.2021
\end{flushright}
\bigskip

%-------------------------------
%	CONTENTS
%-------------------------------

\newtheorem*{task}{Задача}
\begin{task}
Найти общее решение дифференциального уравнения
\begin{equation*}
    x^2y''(x)+xy'(x)+4y(x)=10x.
\end{equation*}
\end{task}

\noindent\textbf{Решение.} 
Это уравнение является частным случаем дифференциального уравнения Эйлера. Его можно свести к линейному уравнению с постоянными коэффициентами с помощью замены
\begin{equation*}
    x = x(t) = e^t.
\end{equation*}
Теперь считаем $t$ независимой переменной; для простоты можно ввести функцию
\begin{equation*}
    v(t) = y(e^t) = y(x).
\end{equation*}
Тогда её производные выписываются следующим образом:
\begin{gather*}
    \dot v(t) = \frac{d}{dt} y(e^t) = e^t y'(e^t) = xy'(x),\\
    \ddot v(t) = \frac{d}{dt} v'(t) = \frac{d}{dt} \left(e^t y'(e^t)\right) = e^{2t}y''(e^t) + e^t y'(e^t) = x^2y''(x)+xy'(x) = x^2y''(x)+\dot v(t),\\
    \Rightarrow x^2y''(x) = \ddot v(t) - \dot v(t).
\end{gather*}
Подставляем эти выражения в исходное уравнение:
\begin{equation}\label{non-uniform}
    \ddot v(t) + 4v(t) = 10e^t.
\end{equation}
Получилось линейное неоднородное уравнение с постоянными коэффициентами. Сначала получим решение однородного уравнения
\begin{equation*}
    \ddot v(t) + 4v(t) = 0.
\end{equation*}
Его общее решение:
\begin{equation}\label{uniform}
    v_0(t) = C_1\cos{2t} + C_2\sin{2t}.
\end{equation}
Как известно, общее решение неоднородного уравнения (\ref{non-uniform}) представляется как сумма его частного решения и общего решения однородного уравнения (\ref{uniform}). Частное решение уравнения (\ref{non-uniform}) естественно искать в виде $ae^t$, где $a$ -- константа, подлежащая определению. Подставив в уравнение, получим, что $a = 2$. Таким образом,
\begin{equation*}
    v(t) = C_1\cos{2t} + C_2\sin{2t} + 2e^t.
\end{equation*}
Возвращаемся к замене. Отметим, что при замене $x=e^t$ формально $x$ может принимать только положительные значения, но если положить $t = \ln{|x|}$, то следующая формула останется справедливой для всех ненулевых $x$:
\begin{equation*}
    y(x) = C_1\cos{(2\ln{|x|})} + C_2\sin{(2\ln{|x|})} + 2x.
\end{equation*}
\end{document}