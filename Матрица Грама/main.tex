\documentclass[12pt]{article} 
\usepackage[russian]{babel}
\usepackage{amsmath}			% for math formulas
\usepackage{amsthm}
\usepackage{setspace}
\usepackage{amsfonts}			% for math formulas
\usepackage{amssymb}
\usepackage[unicode, pdftex]{hyperref}
\usepackage[left=25mm, top=20mm, right=25mm, bottom=20mm, nohead, nofoot]{geometry}
\pagestyle{empty}
\begin{document}

%-------------------------------
%	TITLE SECTION
%-------------------------------


\begin{flushleft}
\url{https://www.facebook.com/profile.php?id=10000654912}

\url{https://github.com/lovgager/latex}
\end{flushleft}
\hrule 
\begin{flushright}
14.04.2021
\end{flushright}
\bigskip

%-------------------------------
%	CONTENTS
%-------------------------------

\newtheorem*{task}{Задача}
\begin{task}
В двумерном евклидовом пространстве задано скалярное произведение \\ $\langle x,y \rangle = 3x_1y_1+7x_2y_2.$ Для системы векторов $a,b$ матрица Грама равна 
\begin{equation*}
    \begin{pmatrix}
    34 & -39\\
    -39 & 75
    \end{pmatrix}.
\end{equation*}
Найти вектор $a$, если $b = (-2,3)$ и известно, что $a_1 > 0.$
\end{task}

\noindent\textbf{Решение.} 
По определению матрица Грама для системы векторов $a,b$ -- это матрица всевозможных скалярных произведений, а именно:
\begin{equation*}
    \begin{pmatrix}
    \langle a, a \rangle & \langle a, b \rangle \\
    \langle b, a \rangle & \langle b, b \rangle
    \end{pmatrix}.
\end{equation*}
Если $a=(a_1,a_2), \enskip b=(b_1, b_2)$, то, пользуясь определением скалярного произведения (из условия), можем записать:
\begin{gather*}
    \langle a, a \rangle = 3a_1^2+7a_2^2 = 34,\\
    \langle a, b \rangle = 3a_1b_1 + 7a_2b_2 = -6a_1+21a_2=-39.
\end{gather*}
Таким образом, получаем систему из двух уравнений относительно неизвестных $a_1, a_2$. Если выразить $a_2$ из второго уравнения и подставить в первое, получим:
\begin{equation*}
    25a_1^2-52a_1^2-69=0.
\end{equation*}
Это уравнение имеет два корня, но по условию $a_1>0$, поэтому остаётся только корень $a_1=3.$ Из системы находим значение $a_2 = -1.$ 

Отметим, что $\langle b,b\rangle=75$, поэтому если бы в исходной матрице правый нижний элемент не равнялся $75$, то она бы не являлась матрицей Грама системы векторов $a,b$.

\end{document}