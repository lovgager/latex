\documentclass[12pt]{article} 
\usepackage[russian]{babel}
\usepackage{amsmath}			% for math formulas
\usepackage{amsthm}
\usepackage{setspace}
\usepackage{amsfonts}			% for math formulas
\usepackage{amssymb}
\usepackage[unicode, pdftex]{hyperref}
\usepackage[left=25mm, top=20mm, right=25mm, bottom=20mm, nohead, nofoot]{geometry}
\pagestyle{empty}
\begin{document}

%-------------------------------
%	TITLE SECTION
%-------------------------------


\begin{flushleft}
\url{https://www.facebook.com/profile.php?id=10000654912}

\url{https://github.com/lovgager/latex}
\end{flushleft}
\hrule 
\begin{flushright}
07.04.2021
\end{flushright}
\bigskip

%-------------------------------
%	CONTENTS
%-------------------------------

\newtheorem*{task}{Задача}
\begin{task}
В связном планарном графе $G$ степень любой вершины равна трём, а при укладке его на плоскости любая грань (включая внешнюю) окружена циклом длины 5. Найти число вершин и число рёбер в этом графе G.
\end{task}

\noindent\textbf{Решение.} 
Пусть $p$ -- число вершин графа $G$;  $q$ -- число рёбер; $r$ -- число граней. Для степеней вершин справедлива формула
\begin{equation}\label{deg}
    \sum\limits_{v\in V} d(v) = 2q.
\end{equation}
Здесь $V$ -- множество всех вершин графа $G$, а $d(v)$ -- степень вершины $v$, то есть число рёбер, исходящих из неё. Действительно, если для каждой вершины посчитать число исходящих из неё ребёр и все эти числа сложить, то получится удвоенное число всех рёбер графа, поскольку каждое ребро считается дважды (так как соединяет ровно две вершины).

Для планарных графов (то есть для таких, которые допускают укладку на плоскости без пересечений рёбер) справедлива формула Эйлера:
\begin{equation}\label{euler}
    p - q + r = 2.
\end{equation}

Кроме того, по условию каждая грань окружена циклом длины 5:
\begin{equation}\label{cycles}
    5r = 2q,
\end{equation}
Каждое ребро считается дважды, так как примыкает ровно к двум граням. 

Таким образом, получаем систему из трёх уравнений (\ref{deg}), (\ref{euler}), (\ref{cycles}):

\begin{equation*}
    \begin{cases}
    3p = 2q,\\
    p-q+r = 2,\\
    5r=2q.
    \end{cases}
\end{equation*}
Решение системы: $p=20,\;q=30,\;r=12.$
\end{document}